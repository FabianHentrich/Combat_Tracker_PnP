\documentclass[a4paper,11pt]{letter}
\usepackage{mathpazo} % Schriftart
\usepackage[a4paper,left=2.5cm,right=2.5cm,top=2.5cm,bottom=2.5cm]{geometry} %Seitenlayout
%\signature{Rektorin Prof. Sylma Vonnex \\ Magica der Fakultät für Elementarmagie \\ Akademie von Morgenstern}
\address{Akademie von Morgenstern}
\date{10. Januar}

\begin{document}

\begin{letter}{Ressourcenausgabe für den genehmigten B-Rang Forschungsauftrag}

% Die Eröffnungslinie für den Brief
\opening{An die Verwaltung,}

% Haupttext des Briefes
die Studierenden erhalten im Rahmen ihrer aktuellen Aufgabe Zugang zu ausgewählten Ressourcen der Akademie. Es handelt sich um vertrauliche Materialien, die in der Ausgabeverwaltung, wenn möglich, neben den üblichen Gegenständen zur Verfügung stehen. Ich vertraue darauf, dass die Studierenden die folgenden Gegenstände in Ihrer Arbeit sorgfältig verwenden:

\begin{itemize}
    \item \textbf{1 x Runenstab der Erleuchtung} 
    \item \textbf{2 x Phönixfeder} 
    \item \textbf{1 x Amulett des Schutzes}
    \item \textbf{1 x Runenstein der Heilung}
\end{itemize}

Zusätzlich erhalten die Studierenden Zugang zu folgenden wissenschaftlichen Abhandlungen: 

\begin{itemize}
    \item \textbf{\textit{Venthir – die verlorene Stadt}}, Arra von Morgenstern et. al.
    \item \textbf{\textit{Lorian Torus – Der Rat}}, Althea von Mirithal.
\end{itemize}

Diese Werke befinden sich in der südlichen Bibliothek, welche nur ausgewählten Studierenden zugänglich ist. Ich vertraue darauf, dass die Studierenden diese Dokumente mit äußerster Diskretion behandeln.

Die Untersuchungen sind von entscheidender Bedeutung für den Erfolg des aktuellen Vorhabens. Ich erwarte, das \textbf{\underline{ALLES}} nach dem Gebrauch in ordnungsgemäßen Zustand zurückgegeben wird.

Start des Auftrags ist der 11. Januar 8 Uhr.

Viel Erfolg.
\vspace{3cm}

\begin{center}
    Rektorin Prof. Sylma Vonnex \\ 
    Magica der Fakultät für Elementarmagie \\ 
    Akademie von Morgenstern
\end{center}
\end{letter}

\end{document}
